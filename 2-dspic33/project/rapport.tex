\documentclass[a4paper]{article}

\usepackage[english]{babel}
\usepackage[utf8]{inputenc}
\usepackage{amsfonts}

\usepackage{amsmath}
\usepackage{graphicx}
\usepackage[colorinlistoftodos]{todonotes}

\title{dsPIC33: Exercise}
\author{Pierre Gerard, Julian Schrembi, Francois Schiltz}

\setlength{\parindent}{0pt}

\begin{document}
\maketitle

\section{Question 1}

\includegraphics[scale=1]{files/elect-01.png} 

\section{Question 2}

\subsection{Computation of the value m}


\verb|INT16U unsigned16temp;|

\verb|unsigned16temp = (k * n);|

\verb|m = (INT8U) (unsigned16temp / (INT16U) N);|
\\

If we directly do the computation $ m = k*n/N $ we would get a wrong result. In fact the multiplication would be too big and some of the result could overflow and be wrong. To solve that problem, we use a intermediate variable of 16bit to store the result of the multiplication. We then do the division and then cast it back to 8 bits. The cast should be correct because the result of the division (m) should stands on 8 bits if the inputs are correct.

\subsection{Computation of the real part}

\verb|signed16temp_r = (INT16S)input[n] * (INT16S) en_r[m];|
\\
First we cast the two 8 bits fixed point integer to two 16 bits fixed point integer and then do the multiplication as shown in the $example2.c$. We put the result in a 16bits integer so the result would be correct.
\\
\verb|signed8temp = (signed16temp_r+128) >> 8;|
\\
Then we are required to put the result in a 8 bits register, so to get better result we first round it and then do the shift to only keep the most significant bits.
\\
\verb|signed16temp_r = (INT16S) signed8temp + (INT16S) output_r[k];|
\\
At last we add the element of the sum to a temporary 16 bits register. We use a 16 bits temporary variable because the addition of two number on 8 bits could overflow.

\subsection{Computation of the imaginary part}

Actually it is the same as the real part with other names for variables.


\section{Question 3}

\subsection{Computation of the value m}

\verb|unsigned16temp = (k * n);|

Since we do INT16 = INT8 * INT8 no overflow will occur ( $2{8}*2{8} = 2{16}$ ). There is no information lost so the accuracy is maximum.

\verb|m = (INT8U) (unsigned16temp / (INT16U) N);|

If the input are in there correct range, no overflow will occur because m must be between 0 and 127 included, so it is ok to cast it to a 8 bit integer. Some precision can be lost during the division since the result is a integer all the time. Plus it loose the decimal, so for example $1.9$ will be $1$ if interpreted as an integer instead of 2 for the nearest integer.

\subsection{Computation of the real part}

%TODO

\subsection{Computation of the imaginary part}

%TODO


\verb INT16S = INT8S * INT8S


\section{Question 4}
The more the frequency grows, the nearest the sample are. So that mean that two sample will have almost the same value. From a certain point, we might see that the sample will have the same value.

% TODO est ce qu'ils veulent une valeur précise ?

\end{document}